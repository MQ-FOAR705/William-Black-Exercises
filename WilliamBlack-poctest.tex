\documentclass[12pt]{article}
\usepackage{focusframe}

% BBS Testing in-document 

% \newcommand{\detailtexcount}[1]{%
%   \immediate\write18{texcount -merge -sum -q #1.tex output.bbl > #1.wcdetail }%
%   \verbatiminput{#1.wcdetail}%
% }
 
% \newcommand{\quickwordcount}[1]{%
%   \immediate\write18{texcount -1 -sum -merge -q #1.tex output.bbl > #1-words.sum }%
%   \input{#1-words.sum} words%
% }
 
% \newcommand{\quickcharcount}[1]{%
%   \immediate\write18{texcount -1 -sum -merge -char -q #1.tex output.bbl > #1-chars.sum }%
%   \input{#1-chars.sum} characters (not including spaces)%
% }

\newcommand{\echofromshell}[1]{%
    %Write output of command into file
    \immediate\write18{echo "hello world #1">test.txt}
    %this is going to echo 
    %Write file into TeX    
    \input{test.txt}
}


\newcommand{\texcountnobibsinglefile}[1]{%
    %https://app.uio.no/ifi/texcount/download.php?doc=TeXcount_3_1_1.pdf
    % -1 Same as specifying -brief and -total, and ensures there will only be one line of output. If used with -sum, the output will only be the total number.
    % -merge Merge included files into document (in place).
    % -quiet no extra output
    % not including the bibliography
    \immediate\write18{texcount -brief -total -sum -merge -quiet #1.tex > #1-words.sum }%
    \input{#1-words.sum} words
}

\title{\textbf{Animation Translation of the Japanese Nation}}
\date{Date of last edit: \today}
\author{William Black 42920477}
\setlength{\parindent}{0pt}
\setlength{\parskip}{0.5em}

\begin{document}

\tableofcontents

\textbf{\maketitle}

\frsect{Background \& Literature Review} {0/1200}

\frsubsect{Leading Paragraph}{0/100}

Translation is a tricky business, you see, ever since time immemorial translators have wondered etc. %\lipsum[1] 

\echofromshell{muahahah}
\texcountnobibsinglefile{WilliamBlack-poctest} % it is important to note that it doesn't count words as a function of \lipsum
% %%TC:ignore
% \quickwordcount{WilliamBlack-poctest}

% \quickcharcount{WilliamBlack-poctest}

% \detailtexcount{WilliamBlack-poctest}
% %%TC:endignore

\frsubsect{Theme Background - identify gap}{0/500}
\frsubsubsect{- what's a translation}{}

A translation transforms the devil into el diablo. %\lipsum[1]

\frsubsubsect{- what's equivalence}{}

Equivalence is about getting the word right and also getting the vibe right. %\lipsum[1]

\frsubsubsect{- what's a sign}{}
\frsubsubsect{- what's a connotative sign}{}
\frsubsubsect{- what's the difference between connotative/denotative signs}{}

\frsect{Aims}{0/300}
\frsubsect{Research Aims}{}
\frsubsect{Project Scope}{}
\frsubsect{Limitations}{}

\end{document}