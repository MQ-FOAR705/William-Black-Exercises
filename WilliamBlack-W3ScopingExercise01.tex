\documentclass[12pt]{article}
\usepackage[utf8]{inputenc}
\usepackage{amssymb}
\usepackage{wasysym}
\usepackage[a4paper, margin=2cm]{geometry}

\title{\textbf{Proof of Concept - Scoping Exercise}}
\date{\today}
\author{William Black 42920477}
\setlength{\parindent}{0pt}

\begin{document}
 
\maketitle \textbf{\large{Main jobs involved in writing my research project:}}
\begin{enumerate}

    \item Data collection and transcription of original and translated animated Disney musical song lyrics
    \begin{itemize}
        \renewcommand{\labelitemi}{$\clock$}
        \renewcommand{\labelitemii}{$\cdot$}
        \item Pains:
        \begin{itemize}
            \item Many lyrics sites make the text unclickable so they can't be easily copy-pasted. 
            \item The quality of lyric transcription is user-dependent and must be checked against the song for accuracy.
        \end{itemize}
        \renewcommand{\labelitemi}{$\sun$}
        \item Pain relievers:
        \begin{itemize}
            \item A tool that can extract text from screencaps.
            \item A program that listens to music lyrics and highlights inaccuracies in text. 
        \end{itemize}
        \renewcommand{\labelitemi}{$\smiley$}    
        \item Gains:
        \begin{itemize}
            \item It would be great if there were a lyrics database where all lyrics needed were available in the same format.
        \end{itemize}
        \renewcommand{\labelitemi}{$\blacksmiley$}
        \item Gain creators:
        \begin{itemize}
            \item A program that listens to music and autogenerates lyrics. Or a search engine that could perform multiple searches at once, returning only websites that have a result for each search.
        \end{itemize}
    \end{itemize}
    
    \item Contrastive linguistic comparison of lyrics against translations
    \begin{itemize}
        \renewcommand{\labelitemi}{$\clock$}
        \renewcommand{\labelitemii}{$\cdot$}
        \item Pains:
        \begin{itemize}
            \item Comparing lyrics between original and translation is very difficult without standardised formatting. User-contributed lyrics can vary in how many bars are included in each line. Japanese lyrics occasionally has line breaks happen between syllables of one multisyllabic character.
        \end{itemize}
        \renewcommand{\labelitemi}{$\sun$}
        \item Pain relievers:
        \begin{itemize}
            \item A tool that listens to music and separates lyrics to music bar parameters that I can choose, to prioritise Japanese spacing when necessary.
        \end{itemize}
        \renewcommand{\labelitemi}{$\smiley$}    
        \item Gains:
        \begin{itemize}
            \item It would be great if there was a way to automatically import lyric sets (that were formatted equal bar length) to Excel so I could assign values to line pairs easily while still able to see entirety of lyric lines.
        \end{itemize}
        \renewcommand{\labelitemi}{$\blacksmiley$}
        \item Gain creators:
        \begin{itemize}
            \item A tool that automatically creates tables out of normal text based on line end.
        \end{itemize}
    \end{itemize}
    
    \item Structural linguistic analysis of differences in lyric meaning
    \begin{itemize}
        \renewcommand{\labelitemi}{$\clock$}
        \renewcommand{\labelitemii}{$\cdot$}
        \item Pains:
        \begin{itemize}
            \item Quite a lot of dictionary checking. I'm not a native speaker and I often use a dictionary plugin to hover over the less familiar Japanese characters, but that doesn't work in Word.
            \item Difficult to mark which translated lines are same vs similar vs different in meaning to original lyrics. Changing colour of text or highlight involves time wasted clicking.
        \end{itemize}
        \renewcommand{\labelitemi}{$\sun$}
        \item Pain relievers:
        \begin{itemize}
            \item A way to use my word processor in my browser so the plugin will still work, or a plugin that works with my word processor.
            \item Keyboard shortcuts that can apply different colours quickly.
        \end{itemize}
        \renewcommand{\labelitemi}{$\smiley$}    
        \item Gains:
        \begin{itemize}
            \item It would be great if I was able to leave myself small coloured notes in the lyrics to help me remember unusual readings, double-meanings, etc.
        \end{itemize}
        \renewcommand{\labelitemi}{$\blacksmiley$}
        \item Gain creators:
        \begin{itemize}
            \item A tool that allows me to select text and leave coloured notes at the selection
        \end{itemize}
    \end{itemize}
    
    \item Writing process
    \begin{itemize}
        \renewcommand{\labelitemi}{$\clock$}
        \renewcommand{\labelitemii}{$\cdot$}
        \item Pains:
        \begin{itemize}
            \item Often changing language when writing leads to confused spell check with distracting red lines. 
            \item My method of staying focused requires me to know how many words have been written for each half-hour spent writing. I currently select paragraphs to see word count manually, because I have so many notes organising my work that it would significantly affect the total word count.
        \end{itemize}
        \renewcommand{\labelitemi}{$\sun$}
        \item Pain relievers:
        \begin{itemize}
            \item A tool that allows me to select text and pick language from a small group of favourite languages.
            \item A program that automatically counts words written during a certain time. A program that automatically counts words written in black, as opposed to notes, temporary placeholders and bibliographies written in blue, red or grey. 
        \end{itemize}
        \renewcommand{\labelitemi}{$\smiley$}    
        \item Gains:
        \begin{itemize}
            \item It would be great if there was an \emph{invisible} framework for me to be able to organise headings, automated word counts for sections, and my own writing suggestions greyed out and not visible in the final product, that I could then create as a template for other sections.
        \end{itemize}
        \renewcommand{\labelitemi}{$\blacksmiley$}
        \item Gain creators:
        \begin{itemize}
            \item An invisible framework where I am able to organise headings, automated word counts for sections, and my own writing suggestions greyed out and not visible in the final product.
        \end{itemize}
    \end{itemize}
    
    \item Create bibliography to support claims
    \begin{itemize}
        \renewcommand{\labelitemi}{$\clock$}
        \renewcommand{\labelitemii}{$\cdot$}
        \item Pains:
        \begin{itemize}
            \item Keeping a million tabs open with potential references highlighted.
            \item Having to look up how to cite references for unusual situations such as translated texts, new publishings of old books, speeches published by third parties, etc. 
        \end{itemize}
        \renewcommand{\labelitemi}{$\sun$}
        \item Pain relievers:
        \begin{itemize}
            \item A plugin that allows saving a tab as a reference software entry with editable autogenerated reference, with highlighted passage as a note.
            \item A tool that knows the referencing rules better than I do and can generate an inline reference and bibliography reference when I fill out a form that has a field for every possible situation.
        \end{itemize}
        \renewcommand{\labelitemi}{$\smiley$}    
        \item Gains:
        \begin{itemize}
            \item It would be great if hovering over an inline/bibliography citation would highlight the paired bibliography/inline citation.
            \item It would be great if references automatically arranged themselves in alphabetical order, and if I could leave a short note with them so I remember which reference the entry is for.
        \end{itemize}
        \renewcommand{\labelitemi}{$\blacksmiley$}
        \item Gain creators:
        \begin{itemize}
            \item A tool that highlights paired references.
            \item A program that organises, edits, attaches notes to, and copies references to clipboard.
        \end{itemize}
    \end{itemize}
    
\end{enumerate}

\end{document}

% Do not understand at all why the margins in page 2 are different.
    % It was because I still had "letterpaper" set from playing through the tutorial
% Helpful to collapse lists when not writing them
% Helpful to leave common sense gaps in code for easier scanning
% Can't work out how to get a .tex document out of this
    % Ok, the answer was to download the .zip file. 